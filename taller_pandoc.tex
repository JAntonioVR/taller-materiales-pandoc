% Options for packages loaded elsewhere
\PassOptionsToPackage{unicode}{hyperref}
\PassOptionsToPackage{hyphens}{url}
%
\documentclass[
]{article}
\usepackage{amsmath,amssymb}
\usepackage{iftex}
\ifPDFTeX
  \usepackage[T1]{fontenc}
  \usepackage[utf8]{inputenc}
  \usepackage{textcomp} % provide euro and other symbols
\else % if luatex or xetex
  \usepackage{unicode-math} % this also loads fontspec
  \defaultfontfeatures{Scale=MatchLowercase}
  \defaultfontfeatures[\rmfamily]{Ligatures=TeX,Scale=1}
\fi
\usepackage{lmodern}
\ifPDFTeX\else
  % xetex/luatex font selection
\fi
% Use upquote if available, for straight quotes in verbatim environments
\IfFileExists{upquote.sty}{\usepackage{upquote}}{}
\IfFileExists{microtype.sty}{% use microtype if available
  \usepackage[]{microtype}
  \UseMicrotypeSet[protrusion]{basicmath} % disable protrusion for tt fonts
}{}
\makeatletter
\@ifundefined{KOMAClassName}{% if non-KOMA class
  \IfFileExists{parskip.sty}{%
    \usepackage{parskip}
  }{% else
    \setlength{\parindent}{0pt}
    \setlength{\parskip}{6pt plus 2pt minus 1pt}}
}{% if KOMA class
  \KOMAoptions{parskip=half}}
\makeatother
\usepackage{xcolor}
\usepackage[left=3cm,right=3cm,top=2cm,bottom=2cm]{geometry}
\usepackage{color}
\usepackage{fancyvrb}
\newcommand{\VerbBar}{|}
\newcommand{\VERB}{\Verb[commandchars=\\\{\}]}
\DefineVerbatimEnvironment{Highlighting}{Verbatim}{commandchars=\\\{\}}
% Add ',fontsize=\small' for more characters per line
\newenvironment{Shaded}{}{}
\newcommand{\AlertTok}[1]{\textcolor[rgb]{1.00,0.00,0.00}{\textbf{#1}}}
\newcommand{\AnnotationTok}[1]{\textcolor[rgb]{0.38,0.63,0.69}{\textbf{\textit{#1}}}}
\newcommand{\AttributeTok}[1]{\textcolor[rgb]{0.49,0.56,0.16}{#1}}
\newcommand{\BaseNTok}[1]{\textcolor[rgb]{0.25,0.63,0.44}{#1}}
\newcommand{\BuiltInTok}[1]{\textcolor[rgb]{0.00,0.50,0.00}{#1}}
\newcommand{\CharTok}[1]{\textcolor[rgb]{0.25,0.44,0.63}{#1}}
\newcommand{\CommentTok}[1]{\textcolor[rgb]{0.38,0.63,0.69}{\textit{#1}}}
\newcommand{\CommentVarTok}[1]{\textcolor[rgb]{0.38,0.63,0.69}{\textbf{\textit{#1}}}}
\newcommand{\ConstantTok}[1]{\textcolor[rgb]{0.53,0.00,0.00}{#1}}
\newcommand{\ControlFlowTok}[1]{\textcolor[rgb]{0.00,0.44,0.13}{\textbf{#1}}}
\newcommand{\DataTypeTok}[1]{\textcolor[rgb]{0.56,0.13,0.00}{#1}}
\newcommand{\DecValTok}[1]{\textcolor[rgb]{0.25,0.63,0.44}{#1}}
\newcommand{\DocumentationTok}[1]{\textcolor[rgb]{0.73,0.13,0.13}{\textit{#1}}}
\newcommand{\ErrorTok}[1]{\textcolor[rgb]{1.00,0.00,0.00}{\textbf{#1}}}
\newcommand{\ExtensionTok}[1]{#1}
\newcommand{\FloatTok}[1]{\textcolor[rgb]{0.25,0.63,0.44}{#1}}
\newcommand{\FunctionTok}[1]{\textcolor[rgb]{0.02,0.16,0.49}{#1}}
\newcommand{\ImportTok}[1]{\textcolor[rgb]{0.00,0.50,0.00}{\textbf{#1}}}
\newcommand{\InformationTok}[1]{\textcolor[rgb]{0.38,0.63,0.69}{\textbf{\textit{#1}}}}
\newcommand{\KeywordTok}[1]{\textcolor[rgb]{0.00,0.44,0.13}{\textbf{#1}}}
\newcommand{\NormalTok}[1]{#1}
\newcommand{\OperatorTok}[1]{\textcolor[rgb]{0.40,0.40,0.40}{#1}}
\newcommand{\OtherTok}[1]{\textcolor[rgb]{0.00,0.44,0.13}{#1}}
\newcommand{\PreprocessorTok}[1]{\textcolor[rgb]{0.74,0.48,0.00}{#1}}
\newcommand{\RegionMarkerTok}[1]{#1}
\newcommand{\SpecialCharTok}[1]{\textcolor[rgb]{0.25,0.44,0.63}{#1}}
\newcommand{\SpecialStringTok}[1]{\textcolor[rgb]{0.73,0.40,0.53}{#1}}
\newcommand{\StringTok}[1]{\textcolor[rgb]{0.25,0.44,0.63}{#1}}
\newcommand{\VariableTok}[1]{\textcolor[rgb]{0.10,0.09,0.49}{#1}}
\newcommand{\VerbatimStringTok}[1]{\textcolor[rgb]{0.25,0.44,0.63}{#1}}
\newcommand{\WarningTok}[1]{\textcolor[rgb]{0.38,0.63,0.69}{\textbf{\textit{#1}}}}
\usepackage{longtable,booktabs,array}
\usepackage{calc} % for calculating minipage widths
% Correct order of tables after \paragraph or \subparagraph
\usepackage{etoolbox}
\makeatletter
\patchcmd\longtable{\par}{\if@noskipsec\mbox{}\fi\par}{}{}
\makeatother
% Allow footnotes in longtable head/foot
\IfFileExists{footnotehyper.sty}{\usepackage{footnotehyper}}{\usepackage{footnote}}
\makesavenoteenv{longtable}
\ifLuaTeX
  \usepackage{luacolor}
  \usepackage[soul]{lua-ul}
\else
  \usepackage{soul}
\fi
\setlength{\emergencystretch}{3em} % prevent overfull lines
\providecommand{\tightlist}{%
  \setlength{\itemsep}{0pt}\setlength{\parskip}{0pt}}
\setcounter{secnumdepth}{-\maxdimen} % remove section numbering
\ifLuaTeX
\usepackage[bidi=basic]{babel}
\else
\usepackage[bidi=default]{babel}
\fi
\babelprovide[main,import]{spanish}
% get rid of language-specific shorthands (see #6817):
\let\LanguageShortHands\languageshorthands
\def\languageshorthands#1{}
\ifLuaTeX
  \usepackage{selnolig}  % disable illegal ligatures
\fi
\IfFileExists{bookmark.sty}{\usepackage{bookmark}}{\usepackage{hyperref}}
\IfFileExists{xurl.sty}{\usepackage{xurl}}{} % add URL line breaks if available
\urlstyle{same}
\hypersetup{
  pdftitle={Herramientas para la elaboración rápida de materiales docentes},
  pdfauthor={Juan Antonio Villegas},
  pdflang={es-ES},
  hidelinks,
  pdfcreator={LaTeX via pandoc}}

\title{Herramientas para la elaboración rápida de materiales docentes}
\usepackage{etoolbox}
\makeatletter
\providecommand{\subtitle}[1]{% add subtitle to \maketitle
  \apptocmd{\@title}{\par {\large #1 \par}}{}{}
}
\makeatother
\subtitle{Uso de Markdown y Pandoc para la confección de documentos,
presentaciones, y más}
\author{Juan Antonio Villegas}
\date{\today}

\begin{document}
\maketitle
\begin{abstract}
A menudo, como docentes, queremos elaborar material de estudio y/o de
apoyo a la docencia de cara a hacer nuestra asignatura más cómoda de
estudiar y nuestras clases más llevaderas y entretenidas de cara al
alumnado. Sin embargo, muchas veces esta intención se ve mermada por la
poca intuitividad y/o los frecuentes fallos de las herramientas
tradicionales, principalmente en matemáticas, física y similares, donde
es común el uso de LaTeX. En este curso introduciré una utilidad cuya
sintaxis es mucho más simple y produce resultados similares, en
ocasiones hasta mejores: Markdown. Unido al uso de `pandoc', este
lenguaje puede producir presentaciones, documentos, informes, páginas
web\ldots{} todo tipo de recursos con muy poco esfuerzo.
\end{abstract}

\section{Introducción a Markdown}\label{introducciuxf3n-a-markdown}

Markdown es:

\begin{itemize}
\tightlist
\item
  Un lenguaje de programación muy sencillo
\item
  Una interfaz simplificada de HTML
\item
  El paso previo a obtener documentos PDF, \LaTeX o presentaciones.
\end{itemize}

\subsection{Posibles usos de Markdown}\label{posibles-usos-de-markdown}

\begin{itemize}
\tightlist
\item
  Los propios ficheros markdown \texttt{*.md}, si se dispone de un
  renderizador como el de
  \href{https://code.visualstudio.com/}{\texttt{VSCode}} o
  \href{https://typora.io/}{\texttt{Typora}} pueden ser útiles por sí
  mismos.
\item
  Sin embargo, hay que tener cuidado, pues las visualizaciones que
  ofrecen estos programas no siempre corresponden con lo que
  posteriormente podemos obtener con pandoc.
\item
  También se utiliza en celdas de texto en notebooks como los de
  \texttt{Jupyter} o \texttt{RStudio}.
\item
  \textbf{El motivo de este taller}: Como paso previo a una compilación
  con \texttt{pandoc}, mediante la cual nuestro sencillo documento de
  markdown se convertirá en un PDF estilo Latex, en un documento
  \texttt{.tex} (transformando los elementos de sintaxis sencilla de
  markdown en latex), en una presentación, en un documento HTML\ldots{}
\end{itemize}

\subsection{Sintaxis Markdown}\label{sintaxis-markdown}

\subsubsection{Títulos}\label{tuxedtulos}

\begin{verbatim}
# Título 1
## Título 2
### Título 3
#### Título 4
##### Título 5
\end{verbatim}

\subsubsection{Estilos de letra}\label{estilos-de-letra}

\begin{verbatim}
**negrita**
*cursiva*
~~tachada~~
<u>subrayada</u> <!-- Sintaxis HTML -->
\end{verbatim}

\textbf{negrita} \emph{cursiva} \st{tachada} subrayada

\subsubsection{Listas}\label{listas}

\begin{verbatim}
Lista:
* item 1

* item 2

    * item 2.1

    * item 2.2

        * item 2.2.1
\end{verbatim}

Lista:

\begin{itemize}
\item
  item 1
\item
  item 2

  \begin{itemize}
  \item
    item 2.1
  \item
    item 2.2

    \begin{itemize}
    \tightlist
    \item
      item 2.2.1
    \end{itemize}
  \end{itemize}
\end{itemize}

\subsubsection{Tablas}\label{tablas}

\begin{longtable}[]{@{}lll@{}}
\toprule\noalign{}
\textbf{Nombre} & \textbf{Departamento} & \textbf{Cargo} \\
\midrule\noalign{}
\endhead
\bottomrule\noalign{}
\endlastfoot
Juan Antonio & Matemática Aplicada & Predoc \\
Esther & Genética & Titular \\
\end{longtable}

\subsubsection{Bloques de Código}\label{bloques-de-cuxf3digo}

\textbf{Python}

\begin{Shaded}
\begin{Highlighting}[]
\KeywordTok{def}\NormalTok{ suma(a,b):}
    \CommentTok{"""Función suma}
\CommentTok{        Input: Dos números enteros o reales a y b}
\CommentTok{        Output: La suma de los números a y b (a+b)}
\CommentTok{    """}
    \ControlFlowTok{return}\NormalTok{ a }\OperatorTok{+}\NormalTok{ b}

\NormalTok{a }\OperatorTok{=} \DecValTok{1}
\NormalTok{b }\OperatorTok{=} \DecValTok{2}
\BuiltInTok{print}\NormalTok{(}\StringTok{"La suma a + b vale "} \OperatorTok{+} \BuiltInTok{str}\NormalTok{(suma(a,b)))}
\end{Highlighting}
\end{Shaded}

\textbf{C++}

\begin{Shaded}
\begin{Highlighting}[]
\DataTypeTok{float}\NormalTok{ suma}\OperatorTok{(}\DataTypeTok{float}\NormalTok{ a}\OperatorTok{,} \DataTypeTok{float}\NormalTok{ b}\OperatorTok{)} \OperatorTok{\{}
    \ControlFlowTok{return}\NormalTok{ a }\OperatorTok{+}\NormalTok{ b}\OperatorTok{;}
\OperatorTok{\}}
\DataTypeTok{int}\NormalTok{ main}\OperatorTok{()} \OperatorTok{\{}
    \DataTypeTok{int}\NormalTok{ a }\OperatorTok{=} \DecValTok{1}\OperatorTok{,}\NormalTok{ b }\OperatorTok{=} \DecValTok{2}\OperatorTok{;}
\NormalTok{    cout }\OperatorTok{\textless{}\textless{}} \StringTok{"La suma a + b vale"} \OperatorTok{\textless{}\textless{}}\NormalTok{ suma}\OperatorTok{(}\NormalTok{a}\OperatorTok{,}\NormalTok{b}\OperatorTok{);}
\OperatorTok{\}}
\end{Highlighting}
\end{Shaded}

\textbf{HTML}

\begin{Shaded}
\begin{Highlighting}[]
\DataTypeTok{\textless{}}\KeywordTok{html}\DataTypeTok{\textgreater{}}
    \DataTypeTok{\textless{}}\KeywordTok{head}\DataTypeTok{\textgreater{}}
\NormalTok{        ...}
    \DataTypeTok{\textless{}/}\KeywordTok{head}\DataTypeTok{\textgreater{}}
    \DataTypeTok{\textless{}}\KeywordTok{body}\DataTypeTok{\textgreater{}}
\NormalTok{        ...}
    \DataTypeTok{\textless{}/}\KeywordTok{body}\DataTypeTok{\textgreater{}}
\DataTypeTok{\textless{}/}\KeywordTok{html}\DataTypeTok{\textgreater{}}
\end{Highlighting}
\end{Shaded}

También es posible incluir \emph{inline} pequeños fragmentos de código
\texttt{de\ esta\ forma}, para ello se usan los caracteres ``. Por
ejemplo: Este fichero se llama \texttt{taller\_pandoc.md}.

\subsubsection{HTML y Latex incrustado}\label{html-y-latex-incrustado}

Podemos utilizar cualquier etiqueta HTML y funcionará de la misma manera
que la sintaxis propia de Markdown, mira la siguiente lista:

Item 1

Item 2 en negrita

Item 3 en azul

También podemos utilizar expresiones de \LaTeX, y si posteriormente
compilamos este documento con \LaTeX podemos también incluir cualquier
elemento propio de la sintaxis \LaTeX, pero con cuidado, pues si
producimos un elemento de otro tipo los elementos propios de latex no
siempre funcionarán.

Por ejemplo, sabemos que, si queremos encontrar las raíces reales (en
\(\mathbb R\)) o complejas (en \(\mathbb C\)) de la ecuación de segundo
grado \(ax^2 + bx + c = 0\), podemos utilizar la expresión
\begin{equation}
\label{eq:segundo-grado}
x_i = \dfrac{-b \pm \sqrt{b^2-4ac}}{2a}, \ \ i=1,2
\end{equation} y tenemos que, si \(x_1\) y \(x_2\) son las soluciones de
\eqref{eq:segundo-grado}, entonces podemos escribir \[
ax^2 + bx + c = a(x-x_1)(x-x_2)
\]

\section{Generación de materiales docentes con
Pandoc}\label{generaciuxf3n-de-materiales-docentes-con-pandoc}

\subsection{¿Qué es y para qué sirve
pandoc?}\label{quuxe9-es-y-para-quuxe9-sirve-pandoc}

Consultar su \href{https://www.pandoc.org/}{página web} y su
\href{https://pandoc.org/MANUAL.html}{documentación}.

\begin{quote}
\emph{``Pandoc: a universal document converter. If you need to convert
files from one markup format into another, pandoc is your swiss-army
knife.''}
\end{quote}

Pandoc es un programa que transforma documentos escritos en un
determinado lenguaje de etiquetas (`markup language') en otro distinto.
Admite una gran cantidad de transformaciones, pero las más utilizadas
son:

\begin{itemize}
\tightlist
\item
  Markdown \(\rightarrow\) \LaTeX
\item
  Markdown \(\rightarrow\) PDF (a través de Latex)
\item
  Markdown \(\rightarrow\) HTML
\end{itemize}

\textbf{Requisitos:} Compilador de Latex
(\href{https://miktex.org/}{miktex}), y por supuesto
\href{https://pandoc.org/installing.html}{pandoc}.

\subsection{Ejemplos de uso}\label{ejemplos-de-uso}

Como hemos dicho anteriormente, a partir de un documento markdown
podemos generar un documento latex, y también un PDF compilado por
latex. Tan solo necesitamos la orden

\begin{Shaded}
\begin{Highlighting}[]
\CommentTok{\# Si se desea un PDF}
\ExtensionTok{pandoc} \AttributeTok{{-}t}\NormalTok{ latex }\AttributeTok{{-}o} \OperatorTok{\textless{}}\NormalTok{nombre\_del\_archivo}\OperatorTok{\textgreater{}}\NormalTok{.pdf }\DataTypeTok{\textbackslash{}}
\NormalTok{    [{-}{-}metadata{-}file=}\OperatorTok{\textless{}}\NormalTok{nombre\_archivo\_conf}\OperatorTok{\textgreater{}}\NormalTok{.yaml] }\DataTypeTok{\textbackslash{}}
    \OperatorTok{\textless{}}\NormalTok{nombre\_del\_archivo}\OperatorTok{\textgreater{}}\NormalTok{.md}

\CommentTok{\# Si se desea un .tex  }
\ExtensionTok{pandoc} \AttributeTok{{-}s} \AttributeTok{{-}t}\NormalTok{ latex }\AttributeTok{{-}o} \OperatorTok{\textless{}}\NormalTok{nombre\_del\_archivo}\OperatorTok{\textgreater{}}\NormalTok{.tex }\DataTypeTok{\textbackslash{}}
\NormalTok{    [{-}{-}metadata{-}file=}\OperatorTok{\textless{}}\NormalTok{nombre\_archivo\_conf}\OperatorTok{\textgreater{}}\NormalTok{.yaml] }\DataTypeTok{\textbackslash{}}
    \OperatorTok{\textless{}}\NormalTok{nombre\_del\_archivo}\OperatorTok{\textgreater{}}\NormalTok{.md}
\end{Highlighting}
\end{Shaded}

\begin{Shaded}
\begin{Highlighting}[]
\CommentTok{\# Si se desea un HTML}
\ExtensionTok{pandoc} \AttributeTok{{-}s} \AttributeTok{{-}o} \OperatorTok{\textless{}}\NormalTok{nombre\_del\_archivo}\OperatorTok{\textgreater{}}\NormalTok{.html }\DataTypeTok{\textbackslash{}}
\NormalTok{    [{-}{-}metadata{-}file=}\OperatorTok{\textless{}}\NormalTok{nombre\_archivo\_conf}\OperatorTok{\textgreater{}}\NormalTok{.yaml] }\DataTypeTok{\textbackslash{}}
    \OperatorTok{\textless{}}\NormalTok{nombre\_del\_archivo}\OperatorTok{\textgreater{}}\NormalTok{.md}
\end{Highlighting}
\end{Shaded}

Además de documentos PDF y latex, también podemos crear presentaciones.

\end{document}
